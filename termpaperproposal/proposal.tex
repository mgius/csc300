% Term paper proposal template - Ilona Sparks
% CSC 300: Professional Responsibilities
% Dr. Clark Turner

% One Column Format
\documentclass[12pt]{article} 

\usepackage{setspace}
\usepackage{url}

%%% PAGE DIMENSIONS
\usepackage{geometry} % to change the page dimensions
\geometry{letterpaper} 


\begin{document}

\title{\vfill Term Paper Proposal: Amazon Removing Books from Publication} %\vfill gives us the black space at the top of the page
\author{
 By Mark Gius \vspace{10pt} \\ 
CSC 300: Professional Responsibilities  \vspace{10pt} \\ 
Dr. Clark Turner \vspace{10pt} \\ 
}
%\date{October 8, 2010} %Or use \Today for today's Date
\date{\today}

\maketitle

\vfill  %in combinaion with \newpage this forces the abstract to the bottom of the page
\begin{abstract}
% TODO: Go over this again
Amazon.com is one of the largest sellers of goods and services on the internet.  In 2003, Amazon acquired Createspace and Booksurge, which provide "On-Demand" publishing services for self-publishing authors. Recently, a number of books published via Createspace have been removed from Amazon's listings and their Kindle book reader devices for reasons of moral yadda yadda.  

Amazon/Createspace are well within their legal rights to refuse publishing/selling of the "banned" books, however they should not have. Amazon should only remove content that is illegal. 

Full Disclosure: I am a current employee of CreateSpace, a fully owned subsidiary of Amazon and the publisher of several books that have been de-listed from Amazon's store.
\end{abstract}

\thispagestyle{empty} %remove page number from title page, but still keep it as pg #1
\newpage


%%%%%%%%%%%%%%%%%%%%
%%% Known Facts  %%%
%%%%%%%%%%%%%%%%%%%%
\section{Facts}
\begin{itemize}
\item Amazon and it's subsidiaries' Member and Content Agreements clearly state that published content can be removed at any time for any reason. \cite{createspaceMemberAgreement} \cite{createspaceContentGuidelines} \cite{AmazonDTPContentGuidelines}

\item Amazon/Createspace, as a private entity, is not bound by the First Amendment.

\item Amazon lists and continues to sell titles that contain content similar to titles that have been de-listed. \cite{AmazonLolitaDTPListing}

\item After de-listing ``The Pedophile's Guide to Love and Pleasure,'' Amazon stated: \\
      ``Amazon believes it is censorship not to sell certain books simply because we or others believe their message is objectionable.  Amazon does not support or promote hatred or criminal acts, however, we do support the right of every individual to make their own purchasing decisions.'' \cite{TechCrunchAmazonCensorship}

\end{itemize}
Known facts that are not disputed that lead to your question. Do not judge these facts or make anything like an argument for an answer in here. Just note the facts that give us the general background and end them with the facts leading to the controversy you are interested in. The reader should naturally be asking the question you'll be asking by that point in your paper. In general, attach your facts to a specific case, the more specific and detailed the facts, the better for your analysis. Cite all facts to their sources. \cite{handout}

%%%%%%%%%%%%%%%%%%%%%%%%%
%%% Research Question %%%
%%%%%%%%%%%%%%%%%%%%%%%%%
\section{Research Question}
Are the recent content removals from Amazon catalogs due to ``offensive'' content ethical?

%%%%%%%%%%%%%%%%%%%%%%%%%
%%% Extant Arguments from External Sources %%%
%%%%%%%%%%%%%%%%%%%%%%%%%
\section{Extant arguments}
Erik Sherman, a writer for BNET, believes that Amazon's current public policy is too vague to be helpful to authors, which will ``allows individual employees to, intentionally or not, create their own versions of corporate strategy.'' He thinks that Amazon's policy on Pornography and Offensive Material should be clarified. \cite{ShermanAmazonExecs}

Selena Kitt, a well known author in the Erotic literature community with about a dozen published novels, had several of her novels de-listed in December.  She is of the opinion that if Amazon published clear guidelines and enforced them consistently, then the issue would be moot.  She is primarily unhappy with the inconsistency of enforcement and vagueness of Amazon's policies.

General opinion of various bloggers and opinion writers is that Amazon should not censure books, but if they choose to remove books from their store it should be the result of a clear and public policy that is consistently enforced. General opinion is also that Amazon is well within legal rights to remove content from their store.

%%%%%%%%%%%%%%%%
%%% Analytic principles %%%
%%%%%%%%%%%%%%%%
\section{Applicable analytic principles}
\begin{itemize}
\item SE Code Section 8.07 says that we should ``Not give unfair treatment to anyone because of any irrelevant prejudices.'' \cite{secode}

\item SE Code Section 8.03 and 3.12 concern the writing of accurate documenation. \cite{secode}

\item ``Our vision is to be earth's most customer centric company; to build a place where people can come to find and discover anything they might want to buy online.'' Amazon Mission Statement can be interpreted as a corporate philosophy, which recent actions seem to contradict. \cite{AmazonIRFAQ}

\item Amazon's Corporate Governance states that the company shall ``Make bold investment decisions in light of long-term leadership considerations rather than short-term profitability considerations.'' \cite{AmazonCorpGovernance}

\item The first amendment to the United States Constitution has often been used to defend controversial material from censorship.  ``Local school boards may not remove books from school library shelves simply because they dislike the ideas contained in those books\ldots'' \cite{BOEvsPico}

\item The moral action is that which results in the greatest utility (happiness).  Utilitarian viewpoint.

\end{itemize}

%%%%%%%%%%%%%%%%
%%% Abstract your Expected Analysis %%%
%%%%%%%%%%%%%%%%
\section{Abstract of Expected Analysis}
Overall, de-listing works causes less ``utility'' than leaving them up.  When a title is de-listed, persons who would have enjoyed the work can no longer do so, and may even become significantly more unhappy.  Those who already read it will be unhappy that the book is no longer available.  The author is robbed of income and prestige, and the publisher also receives less money as a result.  De-listing will make a certain number of persons happier, due to their ``victory'' in the matter, but that happiness is less than the unhappiness caused by the de-listing.  Despite this, in the event of illegal material Amazon would be correct in de-listing a piece of content.

Amazon, although not legally bound by the First Amendment's protections for Speech and Publishing, should strive to uphold those rights regardless of legal obligations.  Amazon should seek to implement the vision of head of Amazon's Kindle business, Russ Grandinetti ``[to make] every book ever written, in any language, in print or out of print, all available within 60 seconds. And we want to make the customer experience great.'' This would make Amazon's policy consistent with SE Code 8.07.  Alternatively, Amazon must provide clear and concise content guidelines and enforce them consistently, which would be consistent with the clear documentation guidelines from SE Code 8.03 and 8.12.

These recent content removals are unethical because they cause a great deal of harm and ....TODO TODO TODO

%cite all the references you want in your annotated bibliography that you cite in the paper
\nocite{*}

\bibliographystyle{IEEEannot}

\bibliography{proposal}

\end{document}
