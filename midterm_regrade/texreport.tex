% CSC 300: Professional Responsibilities
% Dr. Clark Turner

% Two Column Format 
\documentclass[11pt]{article}
%this allows us to specify sections to be single or multi column so that things like title page and table of contents are single column 

\begin{document}
\author{Mark Gius \\
        Section 03}
\date{\today}
\title{}

\maketitle

\newpage

In their paper, \emph{A Rational Design Process: How and Why to Fake It}, David Parnas and John Clements discuss in great detail the process of generating requirements documents for computer software.  They describe a requirements document as ``a place to record the desired behavior of the system as described to us by the user.''  It is important to create a requirements document because ``programmers working on a system are very often not familiar with the application.''  Once finished, this document should be presented to and approved by the users.  Parnas and Clements clearly want us to document the software so that programmers can create what the user wants.

[Section about what might happen if users were not consulted?]

Parnas and Clement's work manifests itself in the ACM Software Engineering Code of conduct, section 3.08.  This section states that a Software Engineer should ``ensure that specifications for software on which they work have been well documented, satisfy the users’ requirements and have the appropriate approvals.''

\newpage

%cite all the references from the bibtex you haven't explicitly cited
\nocite{*}

\bibliographystyle{IEEEannot}

\bibliography{texreport}

\end{document}
