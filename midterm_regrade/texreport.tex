% CSC 300: Professional Responsibilities
% Dr. Clark Turner

% Two Column Format 
\documentclass[11pt]{article}
%this allows us to specify sections to be single or multi column so that things like title page and table of contents are single column 

\begin{document}
\author{Mark Gius \\
        Section 03}
\date{\today}
\title{}

\maketitle

\newpage

In their paper, \emph{A Rational Design Process: How and Why to Fake It}, David Parnas and John Clements discuss in great detail the process of generating requirements documents for computer software.  They describe a requirements document as ``a place to record the desired behavior of the system as described to us by the user.''  It is important to create a requirements document because ``programmers working on a system are very often not familiar with the application.''  Once finished, this document should be presented to and approved by the users.  Parnas and Clements clearly want us to document the expectations for software such that programmers can create exactly what the user wants. 

Consider what might happen if a requirements document were to be skipped, or implemented badly.  Without a clear set of requirements from the user a programmer would not know what it is the user wants.  Unclear requirements force the programmer to make decisions about how software should behave and what it should do.  Every time a programmer decides how a piece of software should behave, they run the risk of making a decision contrary to what the user wants.  As Parnas states, ``having a complete reference on externally visible behavior relieves them of any need to decide what is best for the user.'' By the time the project is complete,  finished product may not resemble what the user had in mind.  This project would be a failure. 

Parnas and Clement's work manifests itself in the ACM Software Engineering Code of conduct, section 3.08.  This section states that a Software Engineer should ``ensure that specifications for software on which they work have been well documented, satisfy the users’ requirements and have the appropriate approvals.''  By following this tenet of the SE Code, a project has a greater chance of producing software that meets the needs of the user and performs as expected.

\newpage

%cite all the references from the bibtex you haven't explicitly cited
\nocite{*}

\bibliographystyle{IEEEannot}

\bibliography{texreport}

\end{document}
