% CSC 300: Professional Responsibilities
% Dr. Clark Turner

% Two Column Format 
\documentclass[11pt]{article}
%this allows us to specify sections to be single or multi column so that things like title page and table of contents are single column 
\usepackage{multicol}  

\usepackage{setspace}
\usepackage{url}

%%% PAGE DIMENSIONS
\usepackage{geometry} % to change the page dimensions
\geometry{letterpaper} 


\begin{document}

\title{\vfill Censorship in Self-Publishing} %\vfill gives us the blank space at the top of the page
\author{
By Mark Gius\vspace{10pt} \\ 
CSC 300: Professional Responsibilities\vspace{10pt} \\ 
Dr. Clark Turner\vspace{10pt} \\ 
}
%\date{October 22, 2010} %Or use \today for today's Date
\date{\today}

\maketitle

\vfill  %in combination with \newpage this forces the abstract to the bottom of the page
\begin{abstract}
Amazon.com is one of the largest sellers of goods and services on the internet.  In 2003, Amazon acquired CreateSpace and BookSurge, which provide ``On-Demand'' publishing services for self-publishing authors. The rise of self publishing services like CreateSpace have lowered the bar for small authors to become published and listed on sites like Amazon. This has allowed more authors to attempt to make a living out of writing.

Recently, a number of books published via CreateSpace and listed on Amazon.com have been removed from Amazon's listings and their Kindle book reader devices.  Public statements from Amazon and personal accounts from authors who have had their books de-listed suggest that books were removed due to their content. 

Is it appropriate for publishers or distributors of books to discriminate what titles they sell due to potentially containing ``offensive content?''  According to the tenets of the ACM Software Engineering Code of Ethics, they should not. 

\vfill

Full Disclosure: I am a current employee of CreateSpace, a fully owned subsidiary of Amazon and the publisher of several books that have been de-listed from Amazon's store. To the best of my knowledge, this paper contains no information that is not available to the public.
\end{abstract}

\thispagestyle{empty} %remove page number from title page
\newpage

%Create a table of contents with all headings of level 3 and above.  
%http://en.wikibooks.org/wiki/LaTeX/Document_Structure#Table_of_contents has info on customizing the table of contents
\thispagestyle{empty}  %Remove page number from TOC
\tableofcontents

\newpage

%start 2 column format
\begin{multicols}{2}
%Start numbering first page of content as page 1
\setcounter{page}{1}
%%%%%%%%%%%%%%%%%%%%
%%% Known Facts  %%%
%%%%%%%%%%%%%%%%%%%%
\section{Facts}

% TODO: Turn into a narrative
\begin{itemize}

\item All of the books that have been de-listed and removed from publication appear to be self-published books through either Amazon DTP (Kindle) or CreateSpace, an Amazon subsidiary.  No books published through more ``traditional'' publishing channels appear to have been removed.  

\item Amazon and its subsidiaries' Member and Content Agreements clearly state that published content can be removed at any time for any reason. \cite{CreateSpaceMemberAgreement,CreateSpaceContentGuidelines,AmazonKDPContentGuidelines}

\item As a private entity, Amazon/CreateSpace are not bound by the First Amendment.

\item Amazon lists and continues to sell titles that contain content similar to titles that have been de-listed. \cite{AmazonLolitaDTPListing}

\item Prior to de-listing ``The Pedophile's Guide to Love and Pleasure,'' Amazon released the following statement: ``Amazon believes it is censorship not to sell certain books simply because we or others believe their message is objectionable.  Amazon does not support or promote hatred or criminal acts, however, we do support the right of every individual to make their own purchasing decisions.'' \cite{TechCrunchAmazonCensorship}

\item Amazon DTP/CreateSpace are not the only self-publisher of books in the market.
   \begin{enumerate}
   
   \item Lulu.com provides self-publishing services that are very similar to that of Amazon/CreateSpace.  Their content policy also prohibits certain types of content, including obscene or offensive material. \cite{LuluMemberAgreement}

   \item Barnes and Noble's PubIt! self-publishing service allows members to publish eBooks through their bookstore, similar to Amazon's DTP Kindle publishing service.  Their content policy also prohibits certain types of content, including obscene or offensive material.  \cite[Select Service Policies, then Content Policy]{PubItContentPolicy}

   \item Blurb provides self-publishing services, and also has a content policy that forbids certain types of content. \cite{BlurbEULA}
   \end{enumerate}

\item As of March 2011 Amazon has not publicly confirmed their reasons for de-listing books. Postings by affected authors state that books were de-listed due to content policy violations \cite{KittSelfPubRevolution}.

\end{itemize}

\subsection{Self Publishing vs. Traditional Publishing}

Because all of the de-listed titles appear to be self-published a short explanation of the major differences between various publishing methods is in order.  I have worked in the Print on Demand publishing industry for over two years, and much of this knowledge is based on my own industry experience.

\subsubsection{Traditional Publishing}

In traditional publishing an author is signed to a publisher.  In exchange, the publisher then offers a number of services to the author.  These services include editing, cover design, art design, and marketing. \cite{WhatPublisherDoes}  At the very end of this process the publisher arranges to have books printed and distributed to retailers. \cite{WhatPublisherDoes}  By the time a book has reached store shelves, the publisher has invested time and money into the author and their work.  The publisher then takes a cut of the book sales profits in exchange for their services.

\subsubsection{Traditional Self-Publishing}

When an author chooses to self-publish a book, all of these tasks are handled by the author himself.  The author may choose to perform some tasks themselves, such as editing and cover design, while hiring companies or individuals to perform marketing or distribution tasks. The author takes on all of the cost and risk associated with publishing the book, but keeps all of the profit.

\subsubsection{Print on Demand Self-Publishing}

A somewhat recent development in the self-publishing world is Print on Demand  (POD) publishing.  Prior to POD, traditional and self-publishers were required to print large quantities of books at a time to reamin cost-effective. This was especially troublesome for self-publishing authors who were forced to order several thousand dollars worth of books \cite{EHowSelfPub} from printers with no guarantee that the book would ever sell.  Print on Demand uses high quality digital printers to make runs of even a single copy economical. Authors are still responsible for editing and proofing their work, but the initial investment in large print runs can be largely avoided.

\subsubsection{eBook Digital Distribution}

The latest publishing method available to authors is digital distribution.  Although digital copies of books have been available for some time through sites such as Project Gutenberg \cite{ProjectGutenberg} the release of popular book readers (commonly referred to as eReaders) such as Amazon's Kindle or Barnes and Noble's Nook have popularized the format.  When an author publishes their book through a digital distribution channel, such as the Kindle Store, they avoid the cost of printing the book entirely and rely on the owner of the digital distribution channel for distribution.  The author is still responsible for tasks such as editing and cover art.  The ``distributor/printer'' has very little overhead as well.  Rather than needing to store a physical product that takes up space in a warehouse or store shelves, eReaders consume digital files that can be stored very cheaply on hard disks.

% Keep this para?
When a POD or digital distribution title is removed from circulation the distributor faces very little loss due to minimal investment in a title.  If a traditionally published title were to be de-listed and banned the publisher could lose a significant amount of money due to the investment in the title. 

%%%%%%%%%%%%%%%%%%%%%%%%%
%%% Research Question %%%
%%%%%%%%%%%%%%%%%%%%%%%%%
\section{Research Question}
Is the Offensive Material section of Amazon's Self Publishing Content Policy ethical?

\subsection{Why is this important?}

% internet low barrier needs citcation
Technologies like Print on Demand and digital distribution lower the barrier to publishing books.  As time goes on more and more authors will publish ever increasing quantities of books.  The lower costs will also make publication of niche topics more cost effective.  These niche topics are likely to offend portions of the population who would rather them not be published.  Much as the internet's low barrier of entry in cost and effort led to an explosion of niche communities and web sites for every conceivable topic \cite{CitationNeeded}, the decreasing barrier to entry and cost of publishing books may lead to an increase in book topic variety.  The rights of authors to publish should be examined early to prevent censoring of material deemed ``offensive.''

%%%%%%%%%%%%%%%%%%%%%%%%%
%%% Extant Arguments from External Sources %%%
%%%%%%%%%%%%%%%%%%%%%%%%%

\section{Arguments}

\subsection{Amazon's actions are Unethical}

\subsubsection{Selena Kitt}

Selena Kitt, a well known author in the Erotic Literature community with about a dozen published novels, had several of her novels de-listed in December.  She is of the opinion that if Amazon published clear guidelines and enforced them consistently, then the issue would be moot.  She is primarily unhappy with the inconsistency of enforcement and vagueness of Amazon's policies. \cite{KittSelfPubRevolution}

\subsubsection{Erik Sherman}

Erik Sherman, a writer for BNET, believes that Amazon's current public policy is too vague to be helpful to authors.  The policy  ``[allows] individual employees to, intentionally or not, create their own versions of corporate strategy.'' \cite{ShermanAmazonExecs} Sherman thinks that Amazon's policy on Pornography and Offensive Material should be clarified.

\section{Amazon's Actions are Ethical}

\subsection{SJ}

SJ, a blogger, purchased one of the de-listed books, ``The Pedophile's Guide to Love and Pleasure,'' so that she could address concerns that all opposed to the book were doing so on false grounds.  She finds the book ``reprehensible,'' and determined that the book in question was a ``how-to-not-get-caught.'' \cite{iasshole}

%%%%%%%%%%%%%%%%
%%% Analysis %%%
%%%%%%%%%%%%%%%%
\section{Analysis}

In the following analysis, there may be some mixing of the company names Amazon and CreateSpace, especially in the context of quoted elements.  CreateSpace is a fully-owned subsidiary of Amazon.  CreateSpace provides self-publishing services for authors.  In general, I will refer to both entities as Amazon for simplicity, unless reason exists to distinguish them.

\subsection{Content Policy}

Customers of Amazon's self-publishing service must agree with several service agreements before they are allowed to publish content.  Depending on their intended medium, they must agree to either CreateSpace's Content Guidelines \cite{CreateSpaceContentGuidelines} or the Kindle Direct Publishing Content Guidelines \cite{AmazonKDPContentGuidelines}.  Both sets of guidelines contain similar prohibitions against certain types of content.  The CreateSpace Content Guidelines are more specific and expansive that those for KDP, so arguments that follow will focus on CreateSpace's content policy.

CreateSpace's policy prohibits, among other things:

\begin{description}
\item[Offensive Material] \hfill \\
      ``What we deem offensive is probably about what you would expect. This includes items such as crime-scene videos, videos of cruelty to animals, and extremely disturbing materials. CreateSpace reserves the right to determine the appropriateness of items sold on our site.'' \cite{CreateSpaceContentGuidelines}

\end{description}

Let's look at the first two sentences in this policy closely.

The first statement, ``What we deem offensive is probably about what you would expect,'' is unclear.  For one, the ``you'' in the sentence is probably intended to refer to the person reading the statement.  However, what one person would expect to be offensive would vary widely from person to person.  In Thai culture, it is considered rude to consume all of the food that is brought to you, because it indicates that the host did not provide enough. \cite{EHowThai}  In Singapore, the possession of chewing gum is a crime, and spitting is also against the law \cite{HotelTravelSingapore}.  As we can see, what is offensive varies by culture, so a policy that expressive offense in terms of a ``you'' will inevitably lead to misunderstandings and confusion.

The next section describes a number of types of content that would be prohibited as ``offensive,'' followed by the catch-all ``extremely disturbing materials.'' Given the context, ``extremely disturbing materials'' could be inferred to mean content similar to those previously listed, but ``disturbing'' is also unclear.  Webster's dictionary defines ``disturbing'' as ``to interfere with.'' \cite{WebsterOnlineDict}  It is unclear what might be interfered with in this context, but Websters lists ``discomfort'' as a synonym for ``disturbing.'' \cite{WebsterOnlineDict}  It should be safe to assume that the intention of the policy is to prevent distribution of content that would discomfort consumers.  Unfortunately, what causes discomfort depends heavily on the viewer and the situation.  For example, a video of cows being slaughtered would cause many people to feel uncomfortable.  However, that same video viewed in a classroom environment could be used to educate students in the methods used to prepare animals for consumption.  The same video would be disturbing or extremely helpful based solely upon the context of the viewer. 

Due to preceding impreciseness, it is possible for enforcers of the content policy to inadvertently apply their own sense of ``disturbing'' or ``offensive,'' resulting in inconsistent application of the policy, as Erik Sherman believes \cite{ShermanAmazonExecs}.  This vague or imprecise wording will be examined in further sections to determine whether the section of the content policy is ethical.

\subsection{Is Amazon's Content Policy Incompatible with ACM SE Code}

The ACM Software Engineering Code of Ethics is a document produced by the ACM/IEEE-CS joint task force on Software Engineering Ethics and Professional Practices in 1999. \cite{SECode} It is a document that is meant to guide the ``ethical behavior of and decisions made by \emph{professional software engineers}, including practitioners, educators, managers, supervisors and \emph{policy makers}.'' \cite[Emphasis Mine]{SECode} The ACM SE Code of Ethics is applicable to the issue of whether or not the content guidelines are ethical because Software Engineers working on Amazon's self publishing system are engineering a system that is expected to behave in a congruous manner with the content guidelines. % TODO: Apparently prove this? Seems reasonable enough

It would be foolish to assume that the content policy was written by software engineers at either Amazon or CreateSpace.  However, regardless of who wrote the content policy, Software Engineers are expected to implement software that will enforce or allow enforcement of the content policy.  The persons who wrote the content policy are clearly ``policy makers,'' and thus the content policy and it's creators should abide by the tenets of the ACM SE Code.

\subsubsection{SE Code Tenet 3.08}

\emph{SE Code 3.08}: ensure that \underline{specifications} for \underline{software} on which they work have been well documented [and] satisfy the \underline{users'} requirements\ldots 

In David Parnas and Paul Clements' paper, ``A Rational Design Process,'' Parnas and Clements discuss the nature of software design. \cite{fakeit} In particular, they discuss the nature of requirements specifications and what should be contained within them.  One of the elements of a proper requirements is a specification for Input/Output Interfaces and Specification Of Output Values.  Their description for the Specification of Output Values is of particular interest, and is defined as follows: ``For each output, a specification of its value in terms of the state and history of the system's environment.'' \cite{fakeit} In other words, based on the specifications every input should have an output that is defined. 

Websters defines ``specification'' as ``a written description of an invention for which a patent is sought.'' \cite{WebsterOnlineDict}  US Title 35, Part II, Chapter 11 \S 112 states the following about the requirements for patents:

{\addtolength{\leftskip}{6mm}

The specification shall contain a written description of the invention, and of the manner and process of making and using it, in such full, clear, concise, and exact terms as to enable any person skilled in the art to which it pertains, or with which it is most nearly connected, to make and use the same, and shall set forth the best mode contemplated by the inventor of carrying out his invention.  \cite{Title35}

}

In other words, specifications are the instructions for how to ``build'' a piece of computer software.

Amazon's self-publishing system accepts inputs from users in the form of the content of their books to be published.  The output of this system is a book that is sold on Amazon's site.  The Content Policy defines the set of acceptable inputs to the self-publishing system.  Content that is in violation of the input constraints should be rejected as invalid input.  The Content Policy for the self-publishing system defines the input specification for the self-publishing system. 

The users of the self-publishing system are authors and other content creators who are interested in self-publishing their book through Amazon's self-publishing software.  For the most part, these end-users are completely unaffected by the content guidelines.  A small portion of the overall population of users will nudge up against the content guidelines, and for this subset of users the self-publishing system may not satisfy the users' requirements.

Based on the contents of the Content Guidelines, it can be deduced that these users who are interested in publishing content that may violate their Content Guidelines are not intended to be users of the self-publishing system.  We have to examine a secondary set of users, those who purchase self-published content, for the self-publishing system to see that this deduction is incorrect.  Consider this quote from Russ Grandinetti, the Vice President of Amazon's Kindle Content division:

{\addtolength{\leftskip}{6mm}

``Our vision is [to make] every book ever written, in any language, in print or out of print, all available within 60 seconds'' \cite{LATimesRussQuote}

}

Amazon's mission statement contains a similar description of Amazon's users:

{\addtolength{\leftskip}{6mm}

``Our vision is to be earth's most customer centric company; to build a place where people can come to find and discover \emph{anything they might want to buy} online.'' \cite[Emphasis Mine]{AmazonIRFAQ}

}

Based on these statements, we can conclude that Amazon's, and their self-publishing system's, ``users'' appear to be the publishers and purchasers of ``every book ever written'' and those looking for ``anything they might want to buy''.

It is important that the self-publishing Content Guidelines be well documented so that users of the self-publishing system can determine whether or not a piece of content is acceptable before they submit it as an input.

We can now rewrite SE Code Tenet 3.08 to apply directly to the situation at hand:

\emph{Substituted SE Code 3.08}: ensure that content guidelines for the self-publishing system on which they work have been well documented and satisfy the requirements of publishers of every book ever written and those looking to buy anything.

The current set of Content Guidelines, especially the guidelines for ``Offensive Content,'' are not well-documented.  As a result, the guideline has been inconsistently enforced.  Amazon removed ``The Pedophile's Guide to Love and Pleasure'' for unspecified reasons, but continues to sell ``Lolita'' \cite{AmazonLolitaDTPListing}, the story of an older man attempting to seduce a 12 year old girl. 

Amazon is also clearly not publishing every book ever written, because some books have been recently de-listed.  Amazon is also not providing ``anything'' that their customers might want to buy, as potential consumers of the de-listed media can no longer purchase content that they are interested in.

In it's current state, Amazon's Content Policy is in conflict with SE Code Tenet 3.08 because the Content Policy does not provide a clear specification for user input and does not meet the requirements of its users.  The Content Policy also conflicts with public statements by Amazon executives and their Mission Statement.

\subsubsection{SE Code Tenet 8.03}

\emph{SE Code 8.03}: \ldots produce \underline{accurate}, informative, and well-written \underline{documentation}. 

According to Thesaurus.com, ``accurate'' can also be expressed as ``definitive,'' meaning ``having its fixed and final form; providing a solution or final answer; satisfying all criteria.'' \cite{Thesaurus, Dictionary}  

Let us think of this content policy to be ``documentation'' for the self-publishing platform, where the content policy defines valid and invalid inputs for publishing content on Amazon.  Websters defines documentation as ``the usually printed instructions, comments, and information for using a particular piece or system of computer software or hardware.'' \cite{WebsterOnlineDict} In other words, documentation will be the set of instructions for the \emph{end-user}, as opposed to specifications, which are instructions for the \emph{software engineer}. 

Websters further defines ``Accurate'' as ``conforming exactly to truth or to a standard.'' \cite{WebsterOnlineDict} In other words, the documentation should indicate or lead to user instructions that are in line with some consistent standard.  If we rewrite the SE Code to more closely match our system, we are left with the following:

\emph{Substituted SE Code 8.03}: Produce conformant to a standard, informative, and well written content policy.

As it stands, Amazon's policy and the enforcement thereof does not appear to be conformant to any discernable standard.  The policy is not informative because there is great confusion amongst end-users as what exactly falls under the ``offensive material'' content policy.  \cite{KittSelfPubRevolution, ShermanAmazonExecs}  The acceptable content policy is incomplete and ambiguous, which leads to user frustration.  Thus, Amazon's content policy is in conflict with SE Code 8.03.

% TODO: Should probably rework this secdtion to be more badass

\subsubsection{SE Code Tenet 8.07}

\emph{SE Code 8.07}: Not give unfair treatment to anyone because of any irrelevant prejudices.

Although the applicability of this tenet should be obvious, it can be rewritten as follows:

\emph{Substituted SE Code 8.07}: [Amazon should] not give unfair treatment to any authors because of any irrelevant prejudices.

It is indisputable that Amazon is treating certain authors differently than others.  As of February 2010, Amazon continues to list Lolita \cite{AmazonLolitaDTPListing}, a book that centers around an older man's attempt to seduce a 12-year old girl, but has removed similar stories by Selina Kitt and other authors. \cite{KittSelfPubRevolution}  The primary difference between the two categories of books is that Lolita was published in 1950 \cite{WikipediaLolita} and is considered to be a literary classic \cite{MLTop100}.  By contrast, author Kitt is considerably less well known, although as of \date{February 24, 2011}, one of Kitt's novels, \emph{Babysitting the Baumgartners}, is the 20th most popular ``erotica'' novel in both print and digital formats on Amazon.com and 1300th overall amongst all digital book content. \cite{AmazonBabysittingListing} Ms. Kitt's works may not be critically rated on the same level as Vladimir Nabokov but notoriety is no basis for selectively listing novels.

Another difference of note between Kitt and Nabokov is that Kitt is a self-publisher through CreateSpace \cite{KittSelfPubRevolution}, while Lolita is published by Vintage, a traditional New York publishing house.  \cite{WorldCatLolita} It is possible that due to Lolita's status as a traditionally published novel with a large publishing firm standing behind it the novel was afforded more protections than Kitt's novel.  

It is important to note that SE Code 8.07 does not state that software engineers must treat all persons equally.  Rather, the code states that they must ``not give unfair treatment'' due to ``irrelevant prejudices.'' Websters defines unfair as ``not equitable in business dealings.'' \cite{WebsterOnlineDict}  Amazon appears to be holding authors to different standards in ``business dealings,'' such that self-published authors find themselves de-listed, where more established authors and publishers do not.  However, the SE Code only applies if unfair treatment occurs due to ``irrelevant prejudices.''  Websters defines prejudice as ``an irrational attitude of hostility directed against an individual, a group, a race, or their supposed characteristics.'' \cite{WebsterOnlineDict} Affected authors appear to fall under the ``group'' of self-publishing erotic authors, specifically those that deal with taboo situations.  De-listing the content of these authors is certainly a hostile action against the livelihood of the authors because if the authors are not listed, they are not making.

The question now depends around whether Amazon has given ``unfair treatment'' due to ``irrelevant prejudices.''  Unfortunately, Amazon has not issued any public statements regarding the de-listing.  Because of this, their rationale and purpose behind the de-listings is unknown, forcing speculation.  It is likely that Amazon has refused to comment publicly to avoid setting precedent for their action and draw more criticism in the future.  In this case, it is clearly their intention to create an environment that allows for inconsistent and arbitrary application of the content policy, which is opposite of the behavior specified by the SE Code.

Amazon's policy and their enforcement of it are in conflict with SE Code Tenet 8.07 because it fosters an environment where authors are treated differently for no rational reason.

\subsection{Is Amazon' Content Policy Censorship} \footnote{DRAFT Note: This section is relatively new and needs cleaning}
Kay Mathieson, a professor of Philosophy and expert on Information Ethics, defines censorship as follows:

``To censor is to restrict or limit access to an expression, portion of an expression, or category of expression, which has been made public by its author, based on the belief that it will be a bad thing if people access the content of that expression.'' \cite{MathiesenCensorship}

% this is flimsy..
In her paper, Mathieson also explains her position on the intention of communication, namely that ``the goal of freedom of expression is not that a speaker gets to speak, but that people are able to communicate with each other.''  In other words, even if a person is not preventing a speaker from speaking, they may be preventing a listener from listening and such an action is still censorship.  Amazon's content policy does not stop authors from creating or speaking their content, but it does block them from distributing it and thus reaching willing participants.  Thus, Amazon's access restricts or limits access to these expressions.

The very act of publishing by an author in effect makes the work public.  The question then falls on whether or not Amazon considers this speech to be a ``bad thing.''  Prior to de-listing ``The Pedophile's Guide to Love and Pleasure'' Amazon issued the following statement:

``Amazon believes it is censorship not to sell certain books simply because we or others believe their message is objectionable.  Amazon does not support or promote hatred or criminal acts, however, we do support the right of every individual to make their own purchasing decisions.'' \cite{TechCrunchAmazonCensorship}

Amazon later de-listed the book, but has refused comment on their reasons for doing so.  As a result, it is impossible to determine if Amazon removed the book from their site based on the belief that ``it will be a bad thing if people access the content.''  Mathieson discusses censorship of content due to ``[moral disapproval] of what persons might \emph{do} with [the] information.'' \cite{MathiesenCensorship}  It is possible that Amazon does not believe a book may be bad in and of itself, but does not wish to contribute to truly illegals acts that may arise from such a book.  Again, without confirmation from Amazon, we cannot know what their motivations were for de-listing the book.

Despite this unknown element, Amazon has committed the \emph{act} of censorship as defined by Mathieson based on unknown motivations.

%TODO: To be continued with why censorship is immoral. Mathieson Paper helpful.  Possibly need to go to oft-cited Cohen text. Even better, bring in one of the ethical systems (Mill's Autonomy mentioned in Mathieson)

%\begin{itemize}
%   \item Should start with a paragraph showing why the SE Code applies to your focus question.
%   \item Sub-headings to delineate your lines of reasoning are required. 
%   \item All arguments must be thoroughly supported by reason and logic. 
%   \item All claims must be supported by reputable primary sources and formal data. 
%   \item SE Code must be central to the argumentation
%   \begin{itemize}
%      \item You should have 2-4 distinct sections of the SE code utilized in your analysis
%      \begin{itemize}
%         \item If section 1 is central to your argument, it is only one of the code sections covered. Do not rely solely on section one. Ex: 1.01-1.04 will not suffice for all of your SE Code based arguments and citations. 
%         \item If discussion about Òpublic goodÓ is used, there must be data to support it. Simply writing Òit benefits the general public because it would make many people happyÓ is insufficient.
%      \end{itemize}
%   \end{itemize}     
%   \item Utilitarian and deontological analysis must be present but not be separate sections 
%   \item Class reading must be referenced as appropriate (at least one paper must be used as the basis of one of the arguments). 
%   \item There should be a clear cohesiveness to the analysis such that each argument logically flows into the next and gently directs the reader toward your conclusion while implicitly providing answers to any doubts they may have through logic and data.
%   \item Opinions > dev/null. \cite{handout}  
%\end{itemize}

%Look at Jason Anderson's how to write a term paper (currently linked as the paper template) for information on how to write this section.  An example of possible sections follows
%\subsection{Why the SE Code Applies}
%\subsection{Argument 1}
%\subsubsection{Code principle 1 that applies}
%\subsubsection{Code principle 2 that applies}
%\subsection{Argument 2}
%\subsubsection{Code principle 1 that applies}
%\subsubsection{Code principle 2 that applies}

%\subsubsection*{}
%Remember to weave the class papers and other ethical systems arguments in with the se code arguments they shouldn't be separate sections.  

%%%%%%%%%%%%%%%%
%%% Conclusion %%%
%%%%%%%%%%%%%%%%
\section{Conclusion}
Amazon's Content Policy for self-published works are unethical because they do not conform to the ACM SE Code of Ethics requirements for documentation, specification and equal treatment, and promotes censorship of works.  To bring their system in line with the SE Code of Ethics, Amazon should clarify their Content Policies such that authors can know prior to submission whether their content meets the guidelines.  To solve the censorship issue, Amazon should allow self-publishers to sell any content that is not strictly illegal.

%end the two column format 
\end{multicols}
\newpage

%cite all the references from the bibtex you haven't explicitly cited
%\nocite{*}

\bibliographystyle{IEEEannot}

\bibliography{texreport}

\end{document}
