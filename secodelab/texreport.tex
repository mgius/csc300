% Term paper proposal template - Ilona Sparks
% CSC 300: Professional Responsibilities
% Dr. Clark Turner

% One Column Format
\documentclass[10pt]{article} 

\usepackage{setspace}
\usepackage{url}
\usepackage{multicol}

%%% PAGE DIMENSIONS
\usepackage{geometry} % to change the page dimensions
\geometry{letterpaper} 


\begin{document}

\title{\vfill SE Code Lab: Amazon Removing Books from Publication} %\vfill gives us the black space at the top of the page
\author{
By Mark Gius \vspace{10pt} \\ 
CSC 300: Professional Responsibilities  \vspace{10pt} \\ 
Dr. Clark Turner \vspace{10pt} \\ 
}
%\date{October 8, 2010} %Or use \Today for today's Date
\date{\today}

\maketitle

\begin{multicols}{2}

%%%%%%%%%%%%%%%%%%%%
%%% Known Facts  %%%
%%%%%%%%%%%%%%%%%%%%
\section{Facts}
\begin{itemize}
\item Amazon and it's subsidiaries' Member and Content Agreements clearly state that published content can be removed at any time for any reason. \cite{createspaceMemberAgreement} \cite{createspaceContentGuidelines} \cite{AmazonDTPContentGuidelines}

\item Amazon/Createspace, as a private entity, is not bound by the First Amendment.

\item Amazon has not publically confirmed their reasons for de-listing books. Postings by affected authors state that books were de-listed for violations of content policy. \cite{KittSelfPubRevolution} 

\item Amazon lists and continues to sell titles that contain content similar to titles that have been de-listed. \cite{AmazonLolitaDTPListing}

\item After de-listing ``The Pedophile's Guide to Love and Pleasure,'' Amazon stated: \\
      ``Amazon believes it is censorship not to sell certain books simply because we or others believe their message is objectionable.  Amazon does not support or promote hatred or criminal acts, however, we do support the right of every individual to make their own purchasing decisions.'' \cite{TechCrunchAmazonCensorship}

\end{itemize}

%%%%%%%%%%%%%%%%%%%%%%%%%
%%% Research Question %%%
%%%%%%%%%%%%%%%%%%%%%%%%%
\section{Research Question}
Are the recent content removals from Amazon catalogs due to ``offensive'' content ethical?

\section{SE Code Analysis}

In the proceeding sections, there may be some mixing of the companies Amazon and Createspace, especially in the context of quoted elements.  Createspace is a fully-owned subsidiary of Amazon, providing self-publishing services for authors.  In general, I will refer to both entities as Amazon for simplicity.

\subsection{Amazon's Content Policy is Unethical because it is vague and applied inconsistently}

As stated in the facts section, Amazon has removed certain books from their bookstore.  Amazon has not publically stated their reason for removing titles from their service, but affected authors have reported their interactions with Amazon customer relations and offered their thoughts on the matter.  Based on information available, Amazon appears to have removed titles based upon their content policy.  Specifically, the policy prohibits, among other things:

\begin{itemize}
\item[Pornography] \hfill \\
      ``Pornography, X-rated movies, home porn, hard-core material that depict graphic sexual acts, and amateur porn''
\item[Offensive Material] \hfill \\
      ``What we deem offensive is probably about what you would expect. This includes items such as crime-scene videos, videos of cruelty to animals, and extremely disturbing materials. CreateSpace reserves the right to determine the appropriateness of items sold on our site.''

\end{itemize}

The first prohibition is fairly clear and specific.  Definitions of pornography can vary, but there are clear precedents in court to help determine what is and is not pornographic \cite{MillerVsCA}.  In any case, the policy is almost in place to restrict content on Amazon's video publishing service. 

The second prohibition is far more broad, which has led to some confusion

%%%%%%%%%%%%%%%%%%%%%%%%%
%%% Extant Arguments from External Sources %%%
%%%%%%%%%%%%%%%%%%%%%%%%%
%\section{Extant arguments}
%Erik Sherman, a writer for BNET, believes that Amazon's current public policy is too vague to be helpful to authors, which will ``[allow] individual employees to, intentionally or not, create their own versions of corporate strategy.'' He thinks that Amazon's policy on Pornography and Offensive Material should be clarified. \cite{ShermanAmazonExecs}
%
%Selena Kitt, a well known author in the Erotic literature community with about a dozen published novels, had several of her novels de-listed in December.  She is of the opinion that if Amazon published clear guidelines and enforced them consistently, then the issue would be moot.  She is primarily unhappy with the inconsistency of enforcement and vagueness of Amazon's policies. \cite{KittSelfPubRevolution}
%
%General opinion of various bloggers and opinion writers is that Amazon should not censure books, but if they choose to remove books from their store it should be the result of a clear and public policy that is consistently enforced. General opinion is also that Amazon is well within legal rights to remove content from their store.
%
%%%%%%%%%%%%%%%%
%%% Analytic principles %%%
%%%%%%%%%%%%%%%%
\section{Applicable analytic principles}
\begin{itemize}
\item SE Code Section 8.07 says that we should ``Not give unfair treatment to anyone because of any irrelevant prejudices.'' \cite{secode}

\item SE Code Section 8.03 and 3.12 concern the writing of accurate documenation. \cite{secode}

\item ``Our vision is to be earth's most customer centric company; to build a place where people can come to find and discover anything they might want to buy online.'' Amazon Mission Statement can be interpreted as a corporate philosophy, which recent actions seem to contradict. \cite{AmazonIRFAQ}

\item Amazon's Corporate Governance states that the company shall ``Make bold investment decisions in light of long-term leadership considerations rather than short-term profitability considerations.'' \cite{AmazonCorpGovernance}

\item The first amendment to the United States Constitution has often been used to defend controversial material from censorship.  ``Local school boards may not remove books from school library shelves simply because they dislike the ideas contained in those books\ldots'' \cite{BOEvsPico}

\item The moral action is that which results in the greatest utility (happiness).  Utilitarian viewpoint.

\end{itemize}

%%%%%%%%%%%%%%%%
%%% Abstract your Expected Analysis %%%
%%%%%%%%%%%%%%%%
\section{Abstract of Expected Analysis}
Overall, de-listing works causes less ``utility'' than leaving them up.  When a title is de-listed, persons who would have enjoyed the work can no longer do so, and may even become significantly more unhappy.  Those who already read it will be unhappy that the book is no longer available.  The author is robbed of income and prestige, and the publisher also receives less money as a result.  De-listing will make a certain number of persons happier, due to their ``victory'' in the matter, but that happiness is less than the unhappiness caused by the de-listing.  Despite this, in the event of illegal material Amazon would be correct in de-listing a piece of content.

Amazon, although not legally bound by the First Amendment's protections for Speech and Publishing, should strive to uphold those rights regardless of legal obligations.  Amazon should seek to implement the vision of head of Amazon's Kindle business, Russ Grandinetti ``[to make] every book ever written, in any language, in print or out of print, all available within 60 seconds. And we want to make the customer experience great.'' This would make Amazon's policy consistent with SE Code 8.07.  Alternatively, Amazon must provide clear and concise content guidelines and enforce them consistently, which would be consistent with the clear documentation guidelines from SE Code 8.03 and 8.12.

These recent content removals are unethical because they cause a great deal of harm.  Amazon, as one of the largest and most influential publishers/vendors of books, should be held to an even higher ethical standard because they set trends in the market and often set precedent that other publishers/vendors follow.

\end{multicols}

%cite all the references you want in your annotated bibliography that you cite in the paper
\nocite{*}

\bibliographystyle{IEEEannot}

\bibliography{texreport}

\end{document}
